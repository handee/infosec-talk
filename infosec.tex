\documentclass[xcolor=table]{beamer}

\graphicspath{{figs/}{./figs/}}
\mode<presentation>
{
%  \usetheme{Warsaw}
  \usetheme{Frankfurt}
%  \setbeamercovered{transparent}
  \usecolortheme{crane}
  \setbeamertemplate{navigation symbols}{}
  \setbeamertemplate{section in toc shaded}[default][60]
  \setbeamertemplate{subsection in toc shaded}[default][50]
  \setbeamertemplate{footline}[frame number]
}


\mode<handout>
{
  \beamertemplatesolidbackgroundcolor{black!5}
  \usecolortheme{dove}
%  \usecolortheme{seagull}
}

\usepackage{url}

% Please add the following required packages to your document preamble:
\usepackage{xcolor}
% If you use beamer only pass "xcolor=table" option, i.e. \documentclass[xcolor=table]{beamer}

\title{Information Security -- attacks and defence}

\author[hmd1]{Hannah Dee \\
  \texttt{hmd1@aber.ac.uk}} 
\date{}
\institute[]{U3A Aberystwyth, April 2023\\
  Aberystwyth University, Department of Computer Science}



%\beamerdefaultoverlayspecification{<+->}

\AtBeginSection[]
{
  \begin{frame}<beamer>
    \frametitle{Outline}
    \tableofcontents[currentsection,currentsubsection,hideothersubsections]
  \end{frame}
}

\begin{document}

\begin{frame}
  \titlepage
\end{frame}

%%%%%%%%%%%%%%%%%%%%%%%%%%%%%%%%%%%%%%%%%%%%%%%%%%%%%%%%%%%%%%%%%

\section{Intro}

\begin{frame}{More of our lives are lived online than ever
	before}
	It seems like (some) people now use the internet for pretty much everything
	\begin{itemize}
		\item Banking
		\item Shopping
		\item Socialising 
		\item Filling in tax returns
		\item Watching TV and films 
		\item Romance (?!) 
	\end{itemize}
\end{frame}
\begin{frame}{The Internet solves a lot of problems}
	It also lets us make more mistakes, more quickly, than ever before.

	\vspace{0.5em}

	It exposes us to the mistakes of others on a wider scale than ever before.

	\vspace{0.5em}

	And it opens us up to more scammers, from a wider range of places, than ever before.
\end{frame}
\begin{frame}{Information security: CIA}
	Information security uses the CIA framework:
	\begin{itemize}
		\item Confidentiality: we want things to be visible only to those people allowed to see them
		\item Integrity: we want things to be changeable only by those people allowed to change them 
		\item Availability: we want things to be accessible when we need to access them
	\end{itemize}
	Thinking about our activity online in these terms can clarify lots of things
\end{frame}
\begin{frame}{Cyberattacks}
	\begin{itemize}
		\item Confidentiality - they want to find out something they shouldn't know, like stealing state secrets
		\item Integrity - they want to change something, like your bank balance
		\item Availability - they want to stop something working, for example by taking down a website they don't like
	\end{itemize}
\end{frame}
\section{Protect your data and identity}
\begin{frame}{Identity}
	You can't have CIA without being able to determine the identity of people who are interacting with your systems.

	\vspace{0.5em}

	\begin{itemize}
		\item We all have multiple intersecting identities
			\begin{itemize}
				\item NI number, UTR, bank account number, passport number, email address, facebook id \ldots
			\end{itemize}
		\item We can prove or validate these identities in a host of different ways
			\begin{itemize}
				\item Documents, passports, bank cards, PINs, and probably, several hundred passwords\ldots
			\end{itemize}
	\end{itemize}

\end{frame}

\begin{frame}{Your data is valuable}
	\begin{itemize}
		\item Your identity is valuable, so things which you might use to prove your identity are also valuable
		\item PII (personal identifying information) is worth protecting
	\end{itemize}

	\vspace{0.5em}
	If threat actors can assume your identity, you've got a problem.
\end{frame}
\begin{frame}{Passwords and other authenticators}
	How many passwords do you have?

	\vspace{0.5em}

	How do you choose them?

	\vspace{0.5em}

	How do you remember them?
\end{frame}
\begin{frame}{Passwords and other authenticators}
		Passwords are a flawed system
	\begin{itemize}
		\item We are poor at remembering them
		\item We are bad at choosing them
		\item We manage them badly
		\item They are sometimes leaked in hacking events or through incompetence
	\end{itemize}
\end{frame}
\begin{frame}{Don't make it easy for attackers}
	If a site is compromised, attackers might steal:
	\begin{itemize}
		\item Name, address
		\item Payment data
		\item Usernames 
		\item Passwords 
	\end{itemize}
	If you've re-used the password, any hacker has an easy job.

	\vspace{0.5em}
	
	You can see if you've been in a breach at \url{https://haveibeenpwned.com/}
\end{frame}
	\begin{frame}{Recommendation 1}

		Use a password manager so you don't have to recall 100s of passwords, and you don't re-use passwords. 

		\vspace{0.5em}

		\url{https://www.ncsc.gov.uk/collection/top-tips-for-staying-secure-online/password-managers}
	\end{frame}
\begin{frame}{Multi-factor authentication (MFA / 2FA)}
	What other systems are there?
	\begin{itemize}
		\item Inherence: things you are (face, fingerprint, voice, other biometrics)
		\item Posession: things you have (smartphone, telephone, tokens, badges and cards)
		\item Knowledge: things you remember ('cognitive passwords', pin numbers)
	\end{itemize}
\end{frame}
	\begin{frame}{Recommendation 2}

Turn on multi-factor authentication for anything important. 

		\vspace{0.5em}

		Email, definitely (it's the gateway to everything else)


		\vspace{0.5em}

		\url{https://www.ncsc.gov.uk/collection/top-tips-for-staying-secure-online/activate-2-step-verification-on-your-email}
	\end{frame}

\begin{frame}{Internet and computer systems are under constant attack}
	\begin{itemize}
		\item Software applications are investigated by not only by ``threat actors'' but also by ``bug hunters'' 
		\item When security holes are found, they're reported to the companies in question
		\item This means that often, software is \emph{patched} before any holes are \emph{exploited}
	\end{itemize}

	You can't take advantage of these fixes if you're running old versions
\end{frame}
\begin{frame}{Recommendation 3}

	Install software updates. 

	\vspace{0.5em}

	\url{https://www.ncsc.gov.uk/collection/top-tips-for-staying-secure-online/install-the-latest-software-and-app-updates}

	\vspace{0.5em} 

	(This also applies to ``smart devices'' - watches, Alexa-type things, etc.)

\end{frame}

\section{Social engineering }

\begin{frame}{Social Engineering}

Social engineering is the art of exploiting human psychology,
rather than technical hacking techniques, to gain access to
buildings, systems or data.

Human beings are essentially social creatures.

	\begin{itemize}
		\item We like to help one another.
		\item We generally defer to people higher up in the hierarchy than we are.  
		\item We tend to trust that other people are honest, mean what they say, and are who they say they are
	\end{itemize}
\end{frame}

\begin{frame}{Types of Social Engineering attack}
	\begin{itemize}
		\item Phishing
		\item Pretexting
		\item Baiting
		\item Quid Pro Quo
		\item Tailgating
	\end{itemize}
	Lots of these are as old as time, and aren't so much cyberattacks as con-artistry.
\end{frame}

\begin{frame}{Phishing}
	\begin{itemize}
		\item An email, phonecall or SMS attack
		\item Usually bulk (thousands are sent out - the scammer only needs one or two people to fall for it)
		\item ``You have a delivery'' or maybe ``You have won a prize''
	\end{itemize}
\end{frame}

\begin{frame}{An exercise!}
	Look at the handout, and consider one of the examples
	\begin{enumerate}
		\item What's the scammer exploiting? (Fear? Greed? Desire to help others? Sense of urgency?)
		\item What makes the message suspicious?
		\item What do you think would happen if you fell for it?
	\end{enumerate}

\end{frame}
\begin{frame}{What happens when you follow it up?}
	\begin{itemize}
		\item Might be a site that looks genuine
		\item Might be a site that smells fishy
		\item Might engage you in conversation trying to get you to give over details
		\item Might ask you to pay for something
	\end{itemize}
\end{frame}

\begin{frame}[fragile]
\begin{small}
	\begin{verbatim}
from: prof. chris price <cjp.aberac.uk@gmail.com>
sent: 25 october 2018 11:28:09
to: hannah dee [hmd1]
subject: are you on campus?
 

are you available?
sent from my iphone
	\end{verbatim}
\end{small}
\end{frame}

\begin{frame}[fragile]
\begin{small}
	\begin{verbatim}

--------------------------------------------------------------------
prifysgol aberystwyth www.aber.ac.uk
prifysgol y flwyddyn ar gyfer ansawdd dysgu - the times & the sunday times 2019.

aberystwyth university www.aber.ac.uk
university of the year for teaching quality - the times & the sunday times 2019. 

-- 
personal chair
contact details
• email: cjp@aber.ac.uk
• office: e48, llandinam building
• phone: +44 (0) 1970 622444
• Personal Website: http://users.aber.ac.uk/cjp
	\end{verbatim}
\end{small}
\end{frame}

\begin{frame}[fragile]
\begin{small}
	\begin{verbatim}


On Thu, 25 Oct 2018 at 12:04 PM, Hannah Dee [hmd1]
	<hmd1@aber.ac.uk> wrote:

nope not today, i'm at home. could be in in 15 if needed?


-- 
Dr Hannah Dee, Senior Lecturer,
Computer Science,
Aberystwyth University,
http://users.aber.ac.uk/hmd1

	\end{verbatim}
\end{small}
\end{frame}

\begin{frame}[fragile]
\begin{small}
	\begin{verbatim}

I'm in a meeting right now and that's why I’m contacting you
through here. I should have called you but phone is
not allowed to be used during the meeting. I don't
know when the meeting will be rounding off and I
want you to help me out on something very important
right away.

	\end{verbatim}
\end{small}
\end{frame}


\begin{frame}[fragile]
\begin{small}
	\begin{verbatim}

what would you like me to do?


	\end{verbatim}
\end{small}
\end{frame}


 
\begin{frame}[fragile]
\begin{small}
	\begin{verbatim}

I need you to help me get iTunes Gift cards from the
store, I will reimburse you back when I get back to
the office. I need to send it to someone and it is
very important. I’m still in a meeting and I need to
get it sent right away. It’s for a programming setup
on apple gadgets. The amount I need you to get right
now is £600 I will be reimbursing back to you. I
need physical cards which you are going to get from
the store. When you get them, scratch it and take
pictures of the cards and attach it to this email
then  send it to me here ok.

\end{verbatim}
\end{small}
\end{frame}

\begin{frame}[fragile]
\begin{small}
\begin{verbatim}

sure i can do that. where should i go to get the physical
cards? do they sell them in IBERS Bach?

	\end{verbatim}
\end{small}
\end{frame}

 
\begin{frame}[fragile]
\begin{small}
	\begin{verbatim}
No, go to other stores. 

	\end{verbatim}
\end{small}
\end{frame}

\begin{frame}[fragile]
\begin{small}
	\begin{verbatim}
Which stores would you recommend? I haven't got much time

	\end{verbatim}
\end{small}
\end{frame}

\begin{frame}[fragile]
\begin{small}
	\begin{verbatim}
 
Walmart.
	\end{verbatim}
\end{small}
\end{frame}

\begin{frame}[fragile]
\begin{small}
	\begin{verbatim}

Walmart is a bit challenging - i'm not sure I can get there
and back in time. Might they sell them in Harrods?


	\end{verbatim}
\end{small}
\end{frame}
 
\begin{frame}[fragile]
\begin{small}
	\begin{verbatim}

	Yes, I think you should get them there.

	\end{verbatim}
\end{small}
\end{frame}
\begin{frame}[fragile]
\begin{small}
	\begin{verbatim}


Great I'll just pop to Harrods. Should be back in about 20
	mins. Remind me what you want me to do with these
	cards? 

	\end{verbatim}
\end{small}
\end{frame}

\begin{frame}[fragile]
\begin{small}
	\begin{verbatim}
 
I need you to scratch the cards and attach them to this email.

	\end{verbatim}
\end{small}
\end{frame}
\begin{frame}[fragile]
\begin{small}
	\begin{verbatim}


ok i'm back from harrods with the cards and i have scratched
	them but i am not sure how to attach the cards to
	this email. 

	\end{verbatim}
\end{small}
\end{frame}
\begin{frame}[fragile]
\begin{small}
	\begin{verbatim}
 
Take pictures of the cards and send to this email.

	\end{verbatim}
\end{small}
\end{frame}
\begin{frame}[fragile]
\begin{small}
	\begin{verbatim}

oh, you want pictures of them? 


i have to wait for my camera to charge. 


	\end{verbatim}
\end{small}
\end{frame}
\begin{frame}[fragile]
\begin{small}
	\begin{verbatim}
 
You can write codes out and send here.


	\end{verbatim}
\end{small}
\end{frame}
\begin{frame}[fragile]
\begin{small}
	\begin{verbatim}


you want me to write you some code?


what language?

	\end{verbatim}
\end{small}
\end{frame}
\begin{frame}{Other variants of basically the same scam \ldots}

	\begin{itemize}
		\item ``Hi Mum'' my phone's been stolen so
			I'm borrowing a friends, can I get
			some money to get a new phone?
		\item  I need to get a ticket home, can you \ldots
		\item  I need amazon vouchers \ldots
	\end{itemize}

	\end{frame}
	\begin{frame}{MFA scam}
		\begin{itemize}
				
		\item ``My whatsapp (or some other system) has been hacked they'll
			send out a code but my phone's not
				working can you let me know
			the code?''
		\item You get a reset code from the service in
			question
		\item You send the code to the scammer
		\item That scammer now has your login to the
			service in question
		\end{itemize}
	\end{frame}
\begin{frame}{Consumer level phone scam}
		Phone calls from people who claim to be technicians calling from a `Support Team'. 
	\begin{itemize}
		\item They might even spoof the caller ID {\small(so it looks genuine)}
		\item ``Your computer has a problem''
	\end{itemize}

	To `prove it' they might:
	\begin{itemize}
		\item Show you `error messages' often on fake websites. 
		\item Or full screening your browser with popup messages that won't go away 
		-- locking up the browser {(\small `so computer not work, sir')}.
		\item Or by a opening console app and viewing system logs where there are usually a 
		lot of `warnings' and `errors' and crash logs. 
	\end{itemize}

	{\bf What do they actually want to do?}

\end{frame}

\begin{frame}{Social engineering creates a blended attack}
	What are they trying to (get you to) do? 

	\begin{itemize}
		\item Install a remote control program
		\item Install a keylogger
		\item Install other \textbf{spyware}
		\item Give away a password
		\item Give away data
		\item Hand over money (either for a service,
			or through pretending to be a person
			in need)
	\end{itemize}
		

	
	\vspace{0.5em}

	Rarely (but not never) will you do damage by simply downloading a thing from the Internet. 
\end{frame}

\begin{frame}{Recommendation 4}
	If an email or a call feels suspicious, it probably is.

	\begin{enumerate}
		\item Check independently - can you ask for an official number to call them back on?
		\item Don't let yourself be rushed into anything
		\item Report scam emails
	\end{enumerate}
	\url{https://www.ncsc.gov.uk/collection/phishing-scams/report-scam-email}
	\end{frame}


\section{The future of hacking}
\begin{frame}{On a positive note\ldots}
\end{frame}
\begin{frame}{On a positive note\ldots}

Only kidding. 


	\vspace{0.5em}

	This is not a positive talk.

	\vspace{0.5em}

	\textbf{I see three future trends which will make all this worse}

\end{frame}
\begin{frame}{IoT}
	IoT means ``Internet of Things''
	\begin{itemize}
		\item Internet connected devices aren't just computers
			\begin{itemize}
				\item speaker
				\item thermostat
				\item doorbell
				\item smoke alarm
				\item watch
				\item lightbulbs
				\item car
				\item toaster
			\end{itemize}
		\item This gives a much bigger ``attack surface''
	\end{itemize}

\end{frame}
\begin{frame}{AI}
	How could AI language models create realistic emails?
	\begin{itemize}
		\item AI language models can create realistic emails by using natural language generation algorithms that are trained on large data sets of real emails. These models can analyze patterns in the language and style of real emails, and then generate new emails that are similar in structure and content to the real ones.

		\item Additionally, AI language models can be programmed to personalize emails based on information that is gathered about the target recipient. This information can include the recipient's name, location, job title, or other details that make the email seem more targeted and legitimate.

		\item However, it is important to note that the use of AI to create convincing phishing emails is unethical and illegal. AI should be used in a responsible manner and not for illicit purposes such as cybercrime.
	\end{itemize}
\end{frame}
\begin{frame}{Deepfakes}
	\begin{itemize}
		\item We've already seen (last year, in the USA) voice-based scans which used AI to mimic the vocal patterns of a trusted person.
		\item Scammers can take a video of someone you know, and from that generate a video of that person saying absolutely anything.
		\item In a few years, expect realistic video, real-time. 
	\end{itemize}

	The voice on the phone may sound like someone you know, or the face in the zoom call might look like someone you know\ldots 

	\vspace{0.5em}

	But are they?
\end{frame}
\begin{frame}{My recommendations again:}
	\begin{enumerate}
		\item Use a password manager.
		\item Turn on multi-factor authentication
			for anything important.
		\item Install software updates.
		\item If a message feels suspicious, it
			probably is. Leave it or report it,
			don't answer or follow links.
	\end{enumerate}
\end{frame}

\begin{frame}{Any questions?}
\end{frame}


\end{document}
