\documentclass[xcolor=table]{beamer}

\graphicspath{{figs/}{./figs/}}
\mode<presentation>
{
%  \usetheme{Warsaw}
  \usetheme{Frankfurt}
%  \setbeamercovered{transparent}
  \usecolortheme{crane}
  \setbeamertemplate{navigation symbols}{}
  \setbeamertemplate{section in toc shaded}[default][60]
  \setbeamertemplate{subsection in toc shaded}[default][50]
  \setbeamertemplate{footline}[frame number]
}


\mode<handout>
{
  \beamertemplatesolidbackgroundcolor{black!5}
  \usecolortheme{dove}
%  \usecolortheme{seagull}
}

\usepackage{url}

% Please add the following required packages to your document preamble:
\usepackage{xcolor}
% If you use beamer only pass "xcolor=table" option, i.e. \documentclass[xcolor=table]{beamer}

\title{Information Security -- attacks and defence}

\author[hmd1]{Hannah Dee \\
  \texttt{hmd1@aber.ac.uk}} 
\date{}
\institute[]{U3A Aberystwyth, April 2023\\
  Aberystwyth University, Department of Computer Science}



%\beamerdefaultoverlayspecification{<+->}

\AtBeginSection[]
{
  \begin{frame}<beamer>
    \frametitle{Outline}
    \tableofcontents[currentsection,currentsubsection,hideothersubsections]
  \end{frame}
}

\begin{document}

\begin{frame}
  \titlepage
\end{frame}

%%%%%%%%%%%%%%%%%%%%%%%%%%%%%%%%%%%%%%%%%%%%%%%%%%%%%%%%%%%%%%%%%

\section{Intro}

\begin{frame}{More of our lives are lived online}
	It seems like (some) people now use the internet for pretty much everything
	\begin{itemize}
		\item Banking
		\item Grocery Shopping
		\item Socialising 
		\item Buying holidays
		\item Filling in tax returns
		\item Watching TV and films 
		\item Romance (?!) 
	\end{itemize}
\end{frame}
\begin{frame}{Information security: CIA}
	The predominant conceptual framework in information security is CIA
	\begin{itemize}
		\item Confidentiality: we want things to be visible only to those people allowed to see them
		\item Integrity: we want things to be changeable only by those people allowed to change them 
		\item Availability: we want things to be accessible when we need to access them
	\end{itemize}
	Thinking about our activity online in these terms can clarify lots of things
\end{frame}
\begin{frame}{Identity}
	You can't have CIA without being able to determine the identity of people who are interacting with your systems.
	\begin{itemize}
		\item We all have multiple intersecting identities
			\begin{itemize}
				\item NI number, UTR, bank account number, passport number, email address, facebook id \ldots
			\end{itemize}
		\item We can prove or validate these identities in a host of different ways
			\begin{itemize}
				\item Documents, passports, bank cards, PINs, and probably, several hundred passwords\ldots
			\end{itemize}
	\end{itemize}

\end{frame}

\begin{frame}{Your data is valuable}
	\begin{itemize}
		\item Your identity is valuable, so things which you might use to prove your identity are also valuable
		\item You might not consider things like address, phone number and so on to be sensitive 
		\item PII (personal identifying information) is worth protecting
	\end{itemize}
\end{frame}
\begin{frame}{Passwords and other authenticators}
	\begin{itemize}
		\item Passwords are a flawed system
		\item We are poor at remembering them
		\item We are bad at choosing them
		\item We manage them badly
	\end{itemize}
	What other systems are there?
\end{frame}
\begin{frame}{Multi-factor authentication (MFA / 2FA)}
	\begin{itemize}
		\item Inherence: things you are (face, fingerprint, voice, other biometrics)
		\item Posession: things you have (smartphone, telephone, tokens, badges and cards)
		\item Knowledge: things you remember ('cognitive passwords', pin numbers)
	\end{itemize}
\end{frame}

\begin{frame}{Hackers, crackers, and con artists}
\end{frame}
\section{Technical attacks}

\begin{frame}{Why's it important?}

\end{frame}

\section{Social engineering }

\begin{frame}{Social Engineering}

Social engineering is the art of exploiting human psychology,
rather than technical hacking techniques, to gain access to
buildings, systems or data.

Human beings are essentially social creatures.

	\begin{itemize}
		\item We like to help one another.
		\item We generally defer to people higher up in the hierarchy than we are.  
		\item We tend to trust that other people are honest, mean what they say, and are who they say they are
	\end{itemize}
\end{frame}

\begin{frame}{Types of Social Engineering attack}
	\begin{itemize}
		\item Phishing
		\item Pretexting
		\item Baiting
		\item Quid Pro Quo
		\item Tailgating
	\end{itemize}
	Lots of these are as old as time, and aren't so much cyberattacks as con-artistry.
\end{frame}

\begin{frame}{Phishing}
	\begin{itemize}
		\item An email, phonecall or SMS attack
		\item Usually bulk (thousands are sent out - the scammer only needs one or two people to fall for it)
		\item ``You have a delivery'' or maybe ``You have won a prize''
	\end{itemize}
\end{frame}

\begin{frame}{Pretexting}
	A family of attack types (could be a phishing attack, or an in-person attack) where the attacker tries to establish a rapport
	\begin{itemize}
		\item Might dress as an engineer from a particular company
		\item Might know details that make them appear more trustworthy
		\item Generates a \emph{pretext} which makes them seem plausible
	\end{itemize}


\end{frame}

\begin{frame}{Baiting}
	Enticing people to install something that's not quite what it seems
	\begin{itemize}
		\item CD-ROMS through the post
		\item USB sticks left in public places
		\item Free downloads that have a malicious aspect
	\end{itemize}
\end{frame}
\begin{frame}{Quid-pro-quo}
	Similar to baiting,  however you are getting a service.

	\begin{itemize}
		\item Visa application sites which will carry out things for a (hefty) fee
		\item Passport application checkers which charge hundreds of pounds 
		\item \ldots
	\end{itemize}

	If it's free, there's a chance the price you'll pay is data.

	\vspace{0.5em}


	Some argue that social media is a quid-pro-quo attack on a grand scale \ldots
\end{frame}

\begin{frame}{Tailgating}
	This is very much a physical attack -- and one which organisations have to defend against, more frequently than individuals.

	\vspace{0.5em}

 	This refers to sneaking through door behind a legitimate employee.  

	\vspace{0.5em}

	Hacker accounts feature lots of ``hanging around in smoking areas''

\end{frame}



\begin{frame}{Social engineering creates a blended attack}
	What are they trying to (get you to) do? 

	\begin{itemize}
		\item Install a remote control program
		\item Install a keylogger
		\item Install other \textbf{spyware}
		\item Give away a password
		\item Give away data
	\end{itemize}
	Rarely (but not never) will you do damage by simply downloading a thing from the Internet. 
\end{frame}

\section{The future of hacking}
\begin{frame}{On a positive note\ldots}
\end{frame}
\begin{frame}{On a positive note\ldots}

Only kidding. 


	\vspace{0.5em}

	This is not a positive talk.

	\vspace{0.5em}

	\textbf{I see three future trends which will make all this worse}

\end{frame}
\begin{frame}{IoT}
	IoT means ``Internet of Things''
	\begin{itemize}
		\item Internet connected devices aren't just computers
			\begin{itemize}
				\item speaker
				\item thermostat
				\item doorbell
				\item smoke alarm
				\item watch
				\item lightbulbs
				\item car
				\item toaster
			\end{itemize}
		\item This gives a much bigger ``attack surface''
	\end{itemize}

\end{frame}
\begin{frame}{AI}
	How could AI language models create realistic emails?
	\begin{itemize}
		\item AI language models can create realistic emails by using natural language generation algorithms that are trained on large data sets of real emails. These models can analyze patterns in the language and style of real emails, and then generate new emails that are similar in structure and content to the real ones.

		\item Additionally, AI language models can be programmed to personalize emails based on information that is gathered about the target recipient. This information can include the recipient's name, location, job title, or other details that make the email seem more targeted and legitimate.

		\item However, it is important to note that the use of AI to create convincing phishing emails is unethical and illegal. AI should be used in a responsible manner and not for illicit purposes such as cybercrime.
	\end{itemize}
\end{frame}
\begin{frame}{Deepfakes}
	\begin{itemize}
		\item We've already seen (last year, in the USA) voice-based scans which used AI to mimic the vocal patterns of a trusted person.
		\item Scammers can take a video of someone you know, and from that generate a video of that person saying absolutely anything.
		\item In a few years, expect realistic video, real-time. 
	\end{itemize}

	The voice on the phone may sound like someone you know, or the face in the zoom call might look like someone you know\ldots 

	\vspace{0.5em}

	But are they?
\end{frame}


\begin{frame}{Any questions?}
\end{frame}


\end{document}
